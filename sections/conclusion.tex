\chapter{Conclusion}\label{chapter:conclusion}


\section{Summary}
Short - what was the target problem?


What did you develop?

What were the results?


\section{Contributions}\label{sec:concl}

How did we / do this answer \crefrange{obj:distributed}{obj:experiments}?

for each objective, state clearly how it was solved and how it helps solve the overall research question

combine objectives with contribution list from 1.5

How does this move the world forward?

\section{Future work}\label{sec:fw}

mention new NVMe kernel space driver here

security / safety

disaggregated memory - new interconnects

iommu in tree structures is a challenge - ats? other solutions?


scaling

%Our system effectively makes all
%hosts, including their internal resources (both devices and memory), part of a common PCIe domain.


%With SmartIO, the hard separation between local and remote is blurred, as remote resources can be used as if they were locally installed and with native PCIe performance.


%Using the \gls{sisciapi}, application memory can be exported as \glspl{sharedsegment}, and \glspl{segment} in remote machines can be mapped into a local application process' virtual address space.
%By building on these concepts, our \lgls{apiext}{extension} makes it possible for a device driver implementation to use all the memory \gls{disaggregation} capabilities of \gls{sisci}, while also providing functionality for abstracting away the location of memory resources and resolving addresses between different address spaces.
%
%A device driver implemented using our \gls{apiext} can be agnostic about the underlying \gls{pcie} network topology, as devices may \gls{dma} directly to \glspl{sharedsegment}, regardless of whether they are local or remote.
%
%It is even possible to map device memory of other devices registered with SmartIO, for example devices borrowed using \gls{dl}.
