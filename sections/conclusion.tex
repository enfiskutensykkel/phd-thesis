\chapter{Conclusion}\label{chapter:conclusion}
As distributed and parallel computing applications are becoming increasingly compute-intensive and data-driven, \gls{io} performance demands are ever growing.
%
Computing accelerators (such as \glspl{fpga} and \glspl{gpu}), high-throughput \glspl{nic}, and fast storage devices like \glspl{nvme}, are now commonplace in most modern computer systems.
%
For distributed computing clusters, distributing such \gls{io} resources in a way that maximizes both performance and resource utilization is a challenge.
%
In this dissertation, we have addressed this challenge and present our SmartIO framework for sharing resources between machines connected over \gls{pcie}.
%
SmartIO makes it possible to scale out and use more hardware resources than there are available in a single machine, as  machines can dynamically share their internal \gls{io} resources with other machines in a heterogeneous \gls{pcie} network.
%
Resources in remote machines can be used as if they were locally installed, without requiring any adaption to device drivers or application software and with native \gls{pcie} performance
%
Moreover, SmartIO is also able to combine traditional device \gls{io} with shared-memory capabilities, unlocking a potential by allowing devices to become part of the same global address space as distributed, cluster applications.




\section{Summary}
% - What was the target problem?
Connecting two or more independent computer systems over \gls{pcie} is possible by using \glspl{pcientb}.
%
\Glspl{ntb} have memory address translation capabilities that allow machines to map (parts of) the address space of remote systems, including \gls{ram} and device memory.
%



\glspl{ntb} to share the internal
Leveraging \glspl{ntb} to allow the internal memory and devices of individual computers to be shared with (and used by) remote machines is a challenge, as mapping remote memory resources 
%
However, in order for \glspl{ntb} to be a viable solution for sharing resources among machines connected over \gls{pcie}
% - What did you develop?


% - What were the results?



\section{Contributions}\label{sec:concl}

How did we / do this answer \crefrange{obj:distributed}{obj:experiments}?
For each objective, state clearly how it was solved and how it helps solve the overall research question as well.
%
Also, objectives should be linked to contributions list in 1.5

How does this move the world forward?


% below: how does this move the world forward
%Although many \gls{disaggregation} solutions for sharing resources over a network already exist, these solutions are often inadequate. %% compared to SmartIO
%%
%For instance, \gls{disaggregation} solutions based on \gls{rdma} introduce additional software complexity that leads to a disparity in performance, compared to a local machine using local resources. %%% compared to smartio, ....
%%
%\Gls{disaggregation} solutions based on \gls{pcie}  where the resources that can shared are limited to devices installed in dedicated servers, as these solutions lack the shared-memory capabilities necessary for sharing the inner resources of individual machines.
%

\section{Future work}\label{sec:fw}

mention new NVMe kernel space driver here

security / safety

disaggregated memory - new interconnects

iommu in tree structures is a challenge - ats? other solutions?


scaling

%Our system effectively makes all
%hosts, including their internal resources (both devices and memory), part of a common PCIe domain.


%With SmartIO, the hard separation between local and remote is blurred, as remote resources can be used as if they were locally installed and with native PCIe performance.


%Using the \gls{sisciapi}, application memory can be exported as \glspl{sharedsegment}, and \glspl{segment} in remote machines can be mapped into a local application process' virtual address space.
%By building on these concepts, our \lgls{apiext}{extension} makes it possible for a device driver implementation to use all the memory \gls{disaggregation} capabilities of \gls{sisci}, while also providing functionality for abstracting away the location of memory resources and resolving addresses between different address spaces.
%
%A device driver implemented using our \gls{apiext} can be agnostic about the underlying \gls{pcie} network topology, as devices may \gls{dma} directly to \glspl{sharedsegment}, regardless of whether they are local or remote.
%
%It is even possible to map device memory of other devices registered with SmartIO, for example devices borrowed using \gls{dl}.
