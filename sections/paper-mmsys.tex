\chapter{Efficient Processing of Videos in a Multi-auditory Environment using Device Lending of GPUs}
\label{paper:MMSys}
\paperthumb

\begin{description}
	\item[Authors:]
	Konstantin Pogorelov, Michael Riegler, \textbf{Jonas Markussen}, H{\aa}kon Kvale Stensland,
	P{\aa}l Halvorsen, Carsten Griwodz, Sigrun Losada Eskeland, Thomas de Lange


	\item[Abstract:]
		In this paper, we present a demo that utilizes Device Lending 
		via \lgls{pcie}{PCI Express (PCIe)} in the context of a multi-auditory
		environment. Device Lending is a transparent, low-latency
		cross-machine \lgls{pcie}{PCIe} device sharing mechanism without any
		the need for implementing application-specific distribution
		mechanisms. As workload, we use a computer-aided diagnosis 
		system that is used to automatically find polyps and
		mark them for medical doctors during a colonoscopy. We
		choose this scenario because one of the main requirements
		is to perform the analysis in real-time. The demonstration
		consists of a setup of two computers that demonstrates how
		Device Lending can be used to improve performance, as well
		as its effect of providing the performance needed for 
		real-time feedback. We also present a performance evaluation
		that shows its real-time capabilities of it.


	\item[Candidate's contributions:]
		Markussen discussed and developed the idea for the paper together with Pogorelov and Riegler, where Markussen was responsible for the Device Lending setup and experiments.
		As a demonstration of a medical computational workload utilizing Device~Lending,
		Markussen performed the performance experiment together with Pogorelov.
		Markussen also contributed with text in all sections, and wrote the section on Device~Lending.

	\item[Published in:]
	Proceedings of the International Conference on Multimedia Systems~(MMSys'16),
	May~2016, article~36, pp.~381--386.
	\doi{10.1145/2910017.2910636}

	\item[Contributed to:]
		TODO

\end{description}
