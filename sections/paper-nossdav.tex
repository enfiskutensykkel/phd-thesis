\chapter{Device Lending in PCI Express Networks}
\label{paper:NOSSDAV}
\paperthumb

\begin{description}
	\item[Authors:]
		Lars Bj{\o}rlykke Kristiansen, \textbf{Jonas Markussen}, H{\aa}kon Kvale Stensland,
		Michael Riegler, Hugo Kohmann, Friedrich Seifert, Roy Nordstr{\o}m, Carsten Griwodz, P{\aa}l Halvorsen

	\item[Abstract:]
		The challenge of scaling \lgls{io}{IO} performance of multimedia systems to demands
		of their users has attracted much research.
		A lot of effort has gone into
		development of distributed systems that add little latency and computing overhead.
		For machines in \lgls{pcie}{PCI Express~(PCIe)} clusters,
		we propose Device Lending as a novel solution which works at a system
		level.
		%
		Device Lending achieves low latency and extremely low computing overhead without
		requiring \textit{any} application-specific distribution mechanisms.
		For the application, the remote \lgls{io}{IO} resource appears local.
		In fact, even the drivers of the operating system remain unaware that
		hardware resources are located in remote machines.
		%
		By enabling machines in a \lgls{pcie}{PCIe} cluster to lend a wide variety of hardware, 
		cluster machines can get temporary access to a pool of \lgls{io}{IO} resources. 
		Network cards, \lgls{fpga}{FPGAs}, \lgls{ssd}{SSDs}, and even \lgls{gpu}{GPUs} can easily 
		be shared among computers.
		Our proposed solution, Device Lending, works transparently without requiring any modifications to drivers,
		operating systems or software applications.

	\item[Candidate's contributions:]
	    Based on Kristiansen's initial implementation of Device~Lending, Markussen had several discussions with Kristiansen and contributed to its development through testing and conducting performance benchmarks.
	    %
		Additionally, Markussen was responsible for writing most of the text and organizing the collaboration with all of the authors.
		%
		He designed and performed the performance evaluation of the Device~Lending method, and also implemented the \acrshort{gpu} \acrshort{rdma} benchmark program used for reference.
		

	\item[Published in:]
		Proceedings of the 26th International Workshop on Network and Operating Systems Support for Digital Audio and Video~(NOSSDAV'16),
		May~2016, article~10, pp.~10:1--10:6.
		\doi{10.1145/2910642.2910650}

	\item[Contributed to:]
		TODO

\end{description}
