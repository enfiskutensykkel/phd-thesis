\chapter{SmartIO: Zero-overhead Device Sharing through PCIe Networking}
\label{paper:TOCS-2021}
\paperthumb

\begin{description}
	\item[Authors:]
		\textbf{Jonas Markussen}, Lars Bj{\o}rlykke Kristiansen, P{\aa}l Halvorsen,
		Halvor Kielland-Gyrud, H{\aa}kon Kvale Stensland, Carsten Griwodz

	\item[Abstract:]
		The large variety of compute-heavy and data-driven applications accelerate the need for a distributed
		I/O solution that enables cost-effective scaling of resources between networked hosts. For example,
		in a cluster system, different machines may have various devices available at different times, 
		but moving workloads to remote units over the network is often costly and introduce 
		large overheads compared to accessing local resources. 
		%
		To facilitate I/O disaggregation and device sharing among hosts connected using PCIe non-transparent
		bridges, we present SmartIO. NVMes, GPUs, network adapters, or any other standard PCIe device may be 
		borrowed and accessed directly, as if they were local to the remote machines.
		%
		We provide capabilities beyond existing disaggregation solutions 
		by combining traditional I/O with distributed shared-memory functionality, allowing devices 
		to become part of the same global address space as cluster applications.
		Software is entirely removed from the data path, and simultaneous sharing of a device among 
		application processes running on remote hosts is enabled.
		%
		Our experimental results show that I/O devices can be shared with remote hosts,
		achieving native PCIe performance.
		%
		Thus, compared to existing device distribution mechanisms, SmartIO provides more efficient, low-cost resource
		sharing, increasing the overall system performance.

	\item[Candidate's contributions:]
		Some stuff

	\item[Published in:]
		ACM Transactions on Computer Systems~(TOCS), 
		June~2021, volume~38, issue~1-2, article~2, pp.~2:1--2:78.
		\doi{10.1145/3462545}

	\item[Contributed to:]
		\Cref{obj:main,obj:sub1}

\end{description}
\includearticle[numbers=none]{papers/TOCS-2021}
