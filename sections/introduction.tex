\chapter{Introduction}\label{sec:intro}
\section{Background and motivation}
\gls{pcie}
\gls{gpu}
\gls{fpga}

\section{Problem statement}\label{sec:objectives}
The wide variety of data-driven cluster computing applications today rely on a combination of shared memory, high-volume storage, and hardware accelerators, such as \glspl{gpu} or \glspl{fpga}.
%
However, while moving data efficiently between networked nodes in a cluster has been a research challenge for decades, moving workloads and data to remote units over the network remains a costly operation that introduces large performance overheads compared to accessing local resources.
%
In addition, something something complexity and generality.
%
In this thesis, we
\begin{quote}
    Can we develop a framework for sharing resources in a \gls{pcie}-networked cluster that makes it easier for individual nodes to scale out and use remote resources while maintaining native \gls{pcie} performance.
\end{quote}


%\begin{quote}\bfseries
%    Can we develop a general framework for sharing and distributing resources between nodes in a \gls{pcie}-networked cluster that blurs the distinction between remote and local and offers native \gls{pcie} performance.
    
    %that makes it easier for nodes to scale out in order to increase overall performance in the cluster and 
    
    %seamlessly combines distributed shared-memory functionality with traditional \gls{io}, has native \gls{pcie} performance, and simultaneously makes it easier to scale out and increase overall performance in the cluster system?
%\end{quote}    
    minimizes complexity blurs the distinction 
    
     and,
    by blurring the distinction between remote and local, 
    
    distributing devices and sharing resources 
    
    
	Can we blur out the distinction between local and remote in a \gls{pcie}-cluster by developing a system for sharing and distributing resources
	that allows nodes to share resources with native \gls{pcie} performance.

%This has created a need for an efficient \gls{io} resource sharing solution.

Can a framework be created

	in order to meet latency and throughput requirements of modern, data-driven workloads?
In this thesis, we address this challenge under the following research question:
\begin{quote}\bfseries
	Can we develop a system for sharing and distributing resources in a \gls{pcie}-networked cluster, allowing cluster nodes to share resources with native \gls{pcie} performance,
	in order to meet latency and throughput requirements of modern, data-driven workloads?
\end{quote}

Moving data within a computer cluster is expensive in several ways, and cluster performance is a active area of research. 
In this thesis, we therefore address the resource sharing and IO performance challenges between computing nodes in a cluster, under the following research question: 



\begin{quote}
    Can we develop an easy to use distributed resource sharing system, yet efficient, low-cost, increasing the overall system performance, in a PCIe-based cluster?
\end{quote}


In particular, this research question is broken down to the following objectives:



%the latency and throughput requirements of modern, data-driven workloads, 
In particular, this research questions are broken down to the following objectives:
\begin{description}
	\item[\objective{lending}:] The framework should be able to efficiently and transparently share devices like disks and GPUs between machines.
	\item[\objective{performance}:] The performance of using remote devices in the system should be close to local access and native PCIe performance.
	\item[\objective{experiments}:] To prove real-world deployment capabilities, the system should be tested on realistic and relevant workloads and benchmarks.
	
\end{description}

Then you 

%The main goal of this thesis is to develop a framework that %unifies
%\begin{description}
%	\item[\objective{lending}:] device lending, framework
%	\item[\objective{mdev}:] mdev
%	\item[\objective{api}:] combine host communication and memory disaggregation with i/o, nvme driver
%	\item[\objective{performance}:] native PCIe performance
%	\item[\objective{lol}:]
%	
%\end{description}

\section{Scope and limitations}
% scope
% both physical hosts and virtual hosts
% also shared memory applications
% distributed, unlike existing solutions, true peer-to-peer sharing
% commodity servers, multi platform amd, intel, arm

% performance, how much performance overhead in using remote devices?

% limitations
% safety, security (iommu, encryption)
% must use ntbs (dolphin ntbs), only sisci
% not a finished product (no orchistration sw [yet])
% for linux only?

\section{Research methodology}

\section{Contributions}
% should experiments be central

\section{Outline}
