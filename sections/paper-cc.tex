\chapter{Flexible Device Compositions and Dynamic Resource Sharing in PCIe Interconnected Clusters using Device~Lending}
\label{paper:CC}
\paperthumb

\begin{description}
	\item[Authors:]
		\textbf{Jonas Markussen}, Lars Bj{\o}rlykke Kristiansen, Rune Johan Borgli, H{\aa}kon Kvale Stensland,
		Friedrich Seifert, Michael Riegler, Carsten Griwodz, P{\aa}l Halvorsen

	\item[Abstract:]
		Modern workloads often exceed the processing and \lgls{io}{I/O} capabilities provided by resource virtualization,
		requiring direct access to the physical hardware in order to reduce latency and computing overhead.
		For computers interconnected in a cluser, access to remote hardware resources often requires facilitation
		both in hardware and specialized drivers with virtualization support. This limits the availability of
		resources to specific devices and drivers that are supported by the virtualization technology being used, as
		well as what the interconnection technology supports.
		%
		For \lgls{pcie}{PCI Express (PCIe)} clusters, we have previously proposed Device Lending as a solution for enabling direct
		low latency access to remote devices.
		The method has extremely low computing overhead, and does not
		require any application- or device-specific distribution mechanisms. Any \lgls{pcie}{PCIe} device, such as network cards disks, and
		\lgls{gpu}{GPUs}, can easily be shared among the connected hosts.
		In this work, we have extended our solution with support for a \lgls{vm}{Virtual Machine (VM)} hypervisor.
		Physical remote devices can be  ``passed through'' to \lgls{vm}{VM} guests, enabling direct access to physical resources 
		while still retaining the flexibility of virtualization. Additionally, we have also implemented multi-device
		support, enabling shortest-path peer-to-peer transfers between remote devices residing in different hosts.
		%
		Our experimental results prove that multiple remote devices can be used, achieving bandwidth and latency
		close to native \lgls{pcie}{PCIe}, and without requiring any additional support in device drivers. 
		\lgls{io}{I/O} intensive workloads run seamlessly using both local and remote resources.
		With our added VM and multi-device support, Device Lending offers highly customizable 
		configurations of remote devices that can be dynamically reassigned and shared to optimize resource 
		utilization, thus enabling a flexible composable \lgls{io}{I/O} infrastructure for \lgls{vm}{VMs} as well as bare-metal
		machines.

	\item[Candidate's contributions:]
		This paper is an extension of \cref{paper:SRMPDS}.
		%
		Markussen and Kristiansen developed the idea for a mechanism for detecting the guest-physical memory layout of a \acrshort{vm}, and Markussen
		extended the \acrshort{mdev} implementation with support for this.
		%
		Markussen was responsible for investigating performance bottlenecks and identified several weak areas,
		and he contributed to further improve the overall performance for both Device~Lending and \acrshort{mdev}.
		%
		He also extended the evaluation by designing and performing additional \acrshort{p2p} and \acrshort{vm} performance experiments,
		as well as conducting a new experiment with an \gls{io}-heavy machine~learning workload 
		to demonstrate the improved performance.
		%
		Markussen wrote most of the text for this paper, and also improved and extended the \acrshort{gpu} benchmarking programs used
		for \cref{paper:SRMPDS}.


	\item[Published in:]
		Cluster Computing,
		online~September~2019,
		issue~date~June~2020, 
		volume~23, issue~2, pp.~1211--1234.
		\doi{10.1007/s10586-019-02988-0}

	\item[Contributed to:]
		TODO

\end{description}
