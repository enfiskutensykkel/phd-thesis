
% Some basic computer related abbreviations not necessary to define for first use
\newacronym{bios}{BIOS}{Basic Input/Output System}
\glsunset{bios}

\newdualentry{io}{I/O}{input/output}
{Any interaction with a hardware device}
\glsunset{io}

\newacronym{ram}{RAM}{random access memory}

\newacronym{cpu}{CPU}{central processing unit}

\newacronym[plural={OSes}]{os}{OS}{operating system}

\newacronym{pci}{PCI}{Peripheral Component Interconnect}
\newacronym{pcie}{PCIe}{Peripheral Component Interconnect Express}


\newacronym{rdma}{RDMA}{remote direct memory access}
\newacronym{nvme}{NVMe}{non-volatile memory express storage device}
\newacronym{nvmeof}{NVMe-oF}{NVMe over Fabrics}
\newacronym{ssd}{SSD}{solid-state flash memory storage device}
\newacronym{gpu}{GPU}{graphics processing unit}
\newacronym{fpga}{FPGA}{field-programmable gate array}
\newacronym{nic}{NIC}{network interface card}
\newacronym{pf}{PF}{physical device function}
\newacronym{mriov}{MR-IOV}{Multi-Root I/O Virtualization}
\newacronym{api}{API}{application programming interface}
\newacronym{msi}{MSI}{message-signalled interrupts}
\newacronym{msix}{MSI-X}{extended message-signalled interrupts}
\newacronym{vm}{VM}{virtual machine}
\newacronym{mdev}{MDEV}{Mediated Device Driver}
\newacronym{sisci}{SISCI}{Software Infrastructure Shared-Memory Cluster Interconnect}
\newacronym{fio}{FIO}{Flexible I/O tester}


\newdualentry{sriov}{SR-IOV}{Single-Root I/O Virtualization}
{Allows a single device to virtualize multiple device functions in hardware}

\newdualentry{dma}{DMA}{Direct Memory Access}
{Devices capable of direct memory access can access system memory and even memory regions of other devices}

\newdualentry{ntb}{NTB}{non-transparent bridge}
{A special \glsfmtshort{pcie} device that translates memory transactions between different address domains}

\newdualentry{bar}{BAR}{Base Address Register}
{The start address of a memory region of a device, used synonymously for the device memory region itself}

\newglossaryentry{p2p}{
    name={peer-to-peer},
    description={A \glsfmtshort{pcie} feature allowing two devices to directly transfer data between each other without going through host \glsfmtshort{ram}}
}



\newglossaryentry{disaggregation}{
	name={disaggregation},
	description={Dividing up a resource, such as memory or a device, into smaller components, and making them available to remote units}
}

\newglossaryentry{kernel space}{
    name={kernel space},
    description={Virtual memory and execution privileges suitable for \glsfmtlong{os} kernel and device drivers}
}

\newglossaryentry{userspace}{
    name={user space},
    description={Virtual memory and execution privileges suitable for application software, in contrast to kernel space}
}

\newglossaryentry{passthrough}{
    name={pass-through},
    description={Allowing a physical hardware device to be accessed directly by a \glsfmtlong{vm} guest by using a system's \glsfmtlong{iommu} to map memory for the device}
}

\newglossaryentry{guest}{
    name={virtual machine guest},
    description={A software-emulated computer system},
    text={guest}
}

\newglossaryentry{host}{
    name={virtual machine host},
    description={The physical machine running one or more \glsfmtlongpl{vm}},
    text={host}
}


\newglossaryentry{middleware}{
    name={middleware},
    description={A software service that provides facilitation beyond functionality available from the \glsfmtlong{os}}
}


\newglossaryentry{paravirtualization}{
	name={paravirtualization},
	description={A paravirtualized device relies on facilitation by the hypervisor in order to use host resources}
}

\newglossaryentry{hypervisor}{
	name={hypervisor},
	description={Kernel space software on a \glsfmtlong{vm} host, assisting and facilitating a virtual machine emulator (such as Qemu)}
}


\newdualentry{iommu}{IOMMU}{I/O Memory Management Unit}
{A unit embedded on the \glsfmtlong{cpu} that creates separate virtual address spaces for devices and prevents \glsfmtlong{dma} transactions outside these virtual address spaces}


\newlinkedacronym{hypervisor}{kvm}{KVM}{Linux kernel-based virtual machine hypervisor}

\newlinkedacronym{gls-sriov}{vf}{VF}{virtual device function}

