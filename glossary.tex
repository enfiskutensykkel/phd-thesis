
% Some basic computer related abbreviations not necessary to define for first use
\newacronym{bios}{BIOS}{Basic Input/Output System}
\glsunset{bios}

\newdualentry{io}{I/O}{input/output}
{Any interaction with a hardware device}
\glsunset{io}

\newacronym{ram}{RAM}{random access memory}

\newacronym{cpu}{CPU}{central processing unit}

\newacronym[plural={OSes}]{os}{OS}{operating system}

\newacronym{pci}{PCI}{Peripheral Component Interconnect}
\newacronym{pcie}{PCIe}{Peripheral Component Interconnect Express}


\newacronym{nvme}{NVMe}{non-volatile memory express storage device}
\newacronym{nvmeof}{NVMe-oF}{NVMe over Fabrics}
\newacronym{ssd}{SSD}{solid-state flash memory storage device}
\newacronym{gpu}{GPU}{graphics processing unit}
\newacronym{fpga}{FPGA}{field-programmable gate array}
\newacronym{nic}{NIC}{network interface card}
\newacronym{pf}{PF}{physical device function}
\newacronym{mriov}{MR-IOV}{Multi-Root I/O Virtualization}
\newacronym{api}{API}{application programming interface}
\newacronym{vm}{VM}{virtual machine}
\newacronym{mdev}{MDEV}{Mediated Device Driver}
\newacronym{sisci}{SISCI}{Software Infrastructure Shared-Memory Cluster Interconnect}
\linkedgls{sisci}{sisciapi}{\glsxtrshort{sisci}~\glsxtrshort{api}}
\newacronym{fio}{FIO}{Flexible I/O tester}

\newglossaryentry{lut}{
    name={look-up table},
    description={Address translation table in an \glsabbrlink{pcie} \glsabbrlink{ntb}.},
    type=nolist
}

\newglossaryentry{function}{
    name={device function},
    text={function},
    description={\glsabbrlink{pcie} devices may have one or more functions, each appearing to the system as individual devices with their own set of resources. Also known as a \glsabbrlink{pcie} endpoint.},
    seealso={gls-sriov},
}
\linkedgls{function}{ep}{endpoint}
\linkedgls{function}{devicefunction}{device~function}

\newglossaryentry{cfgspace}{
    name={configuration space},
    description={A set of standard registers that is used to configure a \glsabbrlink{pcie} device, such as reserving \glsabbrlinkpl{bar} and interrupt addresses.}
}

\newdualentry{sriov}{SR-IOV}{Single-Root I/O Virtualization}
{Allows a single device to virtualize multiple device functions in hardware, each virtual function appearing to the system as a real device (function) with its own set of resources.}

\newdualentry{dma}{DMA}{direct memory access}
{Devices capable of direct memory access can access system memory and even \glsabbrlinkpl{bar} of other devices (\glshyperlink{p2p}).}

\newdualentry[seealso={gls-dma}]{rdma}{RDMA}{remote direct memory access}
{Using the network adapter to copy memory directly onto the network, without going through the \glsabbrlink{os} network stack, often used to implement middleware~services.}

\newdualentry{ntb}{NTB}{non-transparent bridge}
{A special \glsabbrlink{pcie} device that translates memory transactions between separate address spaces.}

\newdualentry[seealso={cfgspace}]{bar}{BAR}{Base Address Register}
{Six registers in a device's configuration space containing the start addresses of the memory regions of a device. The term is often used synonymously for a device memory region.}
\linkedgls{bar}{devicebar}{device~\glsentryshort{bar}}

\newglossaryentry{p2p}{
    name={peer-to-peer},
    description={A \glsabbrlink{pcie} feature allowing two devices to directly transfer data between each other without going through system \glsabbrlink{ram}.},
    seealso={gls-dma}
}
\linkedgls{p2p}{p2pdma}{\glsentrytext{p2p}~\glsentryshort{dma}}

\newglossaryentry{disaggregation}{
	name={disaggregation},
	description={Dividing up a resource, such as memory or a device, into smaller components, and making them available to remote units.}
}
\linkedgls{disaggregation}{disaggregated}{disaggregated}
\linkedgls{disaggregation}{disaggregate}{disaggregate}
\linkedgls{disaggregation}{disaggregating}{disaggregating}

\newglossaryentry{kernelspace}{
    name={kernel space},
    description={Virtual memory and execution privileges suitable for \glsabbrlink{os} kernel and device drivers.}
}

\newglossaryentry{multicasting}{
    name={multicast},
    description={A feature supported by some \glsabbrlink{pcie} switch chips that allow data coming in on one switch port to be replicated on all other switch ports.}
}

\newglossaryentry{userspace}{
    name={user space},
    description={Virtual memory and execution privileges suitable for application software, in contrast to \glshyperlink[kernel space]{kernelspace}.}
}

\newglossaryentry{passthrough}{
    name={pass-through},
    description={Allowing a physical hardware device to be accessed directly by a \glsabbrlink{vm} guest by using a system's \glsabbrlink{iommu} to map memory for the device.},
    seealso={hypervisor}
}

\newglossaryentry{guest}{
    name={virtual machine guest},
    description={A computer system emulated in software.},
    text={guest\glsadd{vm}}
}
\linkedgls{guest}{vmguest}{\glsentryshort{vm}~guest}
\linkedgls{vm}{vminstance}{\glsentryshort{vm}~instance}

\newglossaryentry{host}{
    name={virtual machine host},
    description={The physical machine running one or more \glsabbrlinkpl{vm}.},
    text={host},
    seealso={hypervisor}
}
\linkedgls{host}{hostmachine}{host~machine}
\linkedgls{host}{hosting}{hosting}

\newdualentry{mmio}{MMIO}{memory-mapped I/O}
{In \glsabbrlink{pcie}, device memory, such as registers, are mapped to the same address space as \glsabbrlink{ram}, allowing the \glsabbrlink{cpu} to read and write to this memory the same way it would access system~memory.}

\newglossaryentry{middleware}{
    name={middleware},
    description={A software service that provides facilitation beyond functionality available from the \glsabbrlink{os}, often implemented using \glsabbrlink{rdma}.}
}


\newglossaryentry{paravirtualization}{
	name={paravirtualization},
	description={A paravirtualized device relies on facilitation by the \glshyperlink{hypervisor} in order to use host resources.},
    seealso={hypervisor}
}

\newglossaryentry{hypervisor}{
	name={hypervisor},
	description={Kernel space software on a \glshyperlink[host~machine]{host}, assisting and facilitating a virtual machine emulator (such as Qemu) with \glsabbrlink{iommu} mappings.},
    seealso={gls-iommu}
}


\newdualentry{iommu}{IOMMU}{I/O Memory Management Unit}
{A unit embedded on the \glsabbrlink{cpu} that creates separate virtual address spaces for devices and prevents \glsabbrlink{dma} transactions outside these virtual address spaces.}


\newlinkedacronym[name={KVM}]{hypervisor}{kvm}{KVM~hypervisor}{Linux kernel-based virtual machine hypervisor}

\newlinkedacronym{gls-sriov}{vf}{VF}{virtual device function}

\newdualentry{msi}{MSI}{message-signaled interrupts}
{Interrupts in \glsabbrlink{pcie} are implemented as writes to a special memory address that is interpreted by the \glsabbrlink{cpu} to invoke the correct interrupt vector routine.}

%\newlinkedacronym{gls-msi}{msix}{MSI-X}{extended message-signaled interrupts}

\newglossaryentry{borrower}{
    name={borrower},
    seealso={lender},
    description={A computer system using a remote device with SmartIO.}
}

\newglossaryentry{lender}{
    name={lender},
    seealso={borrower},
    description={A computer system that has registered one or more of its internal devices with SmartIO and allowing it to be used by other machines.}
}
