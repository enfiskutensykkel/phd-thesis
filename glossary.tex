\newcommand{\newacronymwithentry}[4]{%
\newglossaryentry{gls-#1}{
name={#2},
description={#4}
}
\newglossaryentry{#1}{
type=\acronymtype,
name={#2},
description={#3, \glsseeformat[Glossary:]{gls-#1}{}},
first={#3~(#2)\glsadd{gls-#1}}
}
}

\newacronym{ntb}{NTB}{non-transparent bridge}
\newacronym{rdma}{RDMA}{remote direct memory access}
\newacronym{io}{I/O}{input/output}
\newacronym{cpu}{CPU}{central processing unit}
\newacronym{os}{OS}{operating system}
\newacronym{vm}{VM}{virtual machine}
\newacronym{iommu}{IOMMU}{I/O memory management unit}
\newacronym{nvme}{NVMe}{Non-Volatile Memory Express}
\newacronym{ssd}{SSD}{solid-state flash memory drive}
\newacronym{gpu}{GPU}{graphics processing unit}
\newacronym{fpga}{FPGA}{field-programmable gate array}
\newacronym{nic}{NIC}{network interface card}
\newacronym{pf}{PF}{physical device function}
\newacronym{vf}{VF}{SR-IOV virtual device function}
\newacronym{pcie}{PCIe}{Peripheral Component Interconnect Express}


\newacronymwithentry{sriov}{SR-IOV}{Single-Root I/O Virtualization}
{Single-root I/O virtualization allows a single device to virtualize multiple device functions in hardware}

\newacronymwithentry{dma}{DMA}{Direct Memory Access}
{Devices capable of direct memory access can access system memory and even memory regions of other devices}

\newacronymwithentry{bar}{BAR}{Base Address Register}
{The start address of a memory region of a device, used synonymously for the device memory region itself}



\newacronym{mriov}{MR-IOV}{Multi-Root I/O Virtualization}
